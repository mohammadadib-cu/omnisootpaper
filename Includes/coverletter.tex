\begin{titlepage}
	\pagestyle{empty} % Suppress page number
	\nolinenumbers
	\null\vspace{-1.0cm}
	\begin{minipage}[c]{0.5\textwidth}
		\includegraphics[width=\linewidth]{Figures/CarletonLogo/culogo.pdf}
	\end{minipage}
	\hspace{1cm} 
	\begin{minipage}[c]{0.5\textwidth}
		\raggedright
		\setstretch{0.9}
		\textbf{Prof. M Reza Kholghy}\\
		Energy and Particle Technology Laboratory\\
		Dept of Mechanical \& Aerospace Engineering\\
		Library Rd, CB 3202, Ottawa, ON, Canada\\
		reza.kholghy@carleton.ca\\
		https://carleton.ca/eptl/
	\end{minipage}
	\vspace{1cm}
	
	\noindent\textbf{To The Editorial Board of Chemical Engineering Journal:}
	\vspace{0.3cm}
	\setstretch{1.4}
	
	We are pleased to submit our manuscript, titled ``Omnisoot: an object-oriented process design package for gas-phase synthesis of carbonaceous nanoparticles," for consideration in the \textbf{Chemical Engineering Journal}.
	
	The formation of carbonaceous nanoparticles lies at the intersection of chemical engineering, aerosol science, and materials research. These particles have significant environmental and industrial impacts. For example, soot is a major contributor to global warming, while Carbon Black is a valuable industrial additive widely used in tire and rubber manufacturing. Modeling their formation, however, remains challenging due to uncertainties in gas-phase chemistry and limited understanding of particle inception and surface growth processes.
	
	In this work, we present \textbf{Omnisoot}, a robust, object-oriented Python package designed to simulate the formation and evolution of carbonaceous nanoparticles using reduced-order reactor models. To the best of our knowledge, this is the first comprehensive modeling framework of its kind. Omnisoot incorporates four widely used inception models and two population balance models, allowing users to track particle evolution from hydrocarbon pyrolysis through precursor formation, particle inception, surface growth, agglomeration, and oxidation. 
	
	We demonstrate the application of the tool through three representative case studies, highlighting fundamental differences in inception model predictions and their sensitivity to temperature and fuel composition. The package is freely available, and we have provided scripts for all simulation results to support reproducibility and facilitate further use by the research community.
	
	We believe this contribution will be of strong interest to readers of the \textbf{Chemical Engineering Journal}, particularly those focused on gas-phase synthesis, nanoparticle modeling, and computational tools for materials design.
	
	
%	We are excited to submit the manuscript ``Omnisoot: an object-oriented package for process design of gas phase synthesis of carbonaceous nanoparticles" for consideration in \textbf{Chemical Engineering Journal}.
%	
%	The formation of carbonaceous nanoparticles is an active research area at the intersection of chemical engineering, aerosol technology and material science.  Different forms of carbonaceous nanoparticles has an enormous footprint in human life. Two prominent instances are soot, which is a strong contributor to global warming, and Carbon Black, which is a valuable additive widely used in tire and rubber industries. The theoretical description and modeling of carbonaceous nanoparticles face many challenges primarily due to uncertainties in the gas-phase chemistry and the lack of understanding in inception and surface growth rates of these particles. 
%	
%	In this submission, we introduce an a robust computational tool for the description of carbonaceous nanoparticles based on reduced-order reactor models. To our best knowledge, this is the first time that such a comprehensive package has been developed. This package hosts four prominent inception models and two population balance models, which enables a detailed picture of carbonaceous nanoparticles from fuel pyrolysis to the production of precursors, inception of new particles, their surface growth, agglomeration, and oxidation. We studied the application of this tool in three use-cases, and explored the fundamental differences between inception model in each case in terms of sensitivity to temperature and composition.
%	
%	This tool has been published as a publicly available Python package. Additionally, we have provided the links to the scripts used for the simulation results presented in the manuscript, so that the readers and reviewers can use the tool to reproduce the results and further explore the 
	
	All the authors have agreed to its publication. Hoping that it will find a good reception, we thank you for your time and consideration. 
	
	\vspace{1cm}
	\setstretch{1}
	\noindent Yours Sincerely,\\
	M. Reza Kholghy\\
	Assistant Professor and Canada Research Chair, Carleton University\\
	Director of the Energy and Particle Technology Laboratory\\
	July 2025
	
	\vfill
	
	
\end{titlepage}
\clearpage