\usepackage{a4}
\usepackage[utf8]{inputenc}
\usepackage{graphicx,caption,subcaption}
\captionsetup[table]{font={stretch=1}}     %% change 1.2 as you like
\captionsetup[figure]{font={stretch=1}} 
\usepackage{verbatim}
\usepackage[total={6in, 9in}, top=1in,left=1.1in]{geometry} 
\usepackage{moreverb}
\usepackage{psfrag}
\usepackage{amsmath,amssymb,amsfonts,mathrsfs,mathcomp}
\usepackage{upquote}
\usepackage{fancyhdr}
\usepackage[numbers]{natbib}

\usepackage{float}
\usepackage{tikz, graphicx}

\usepackage{colortbl}
%\usepackage[dvipsnames]{xcolor}
\usepackage{setspace}
\usepackage{import}

\usepackage[pagewise]{lineno}
\linenumbers

\makeatletter
\newcounter{reaction}
%%% >> for article <<
\renewcommand\thereaction{R\,\arabic{reaction}}
%%% << for article <<
%%% >> for report and book >>
%\renewcommand\thereaction{R\,\thechapter.\arabic{reaction}}
%\@addtoreset{reaction}{chapter}
%%% << for report and book <<
\newcommand\reactiontag%
{\refstepcounter{reaction}\tag{\thereaction}}
\newcommand\reaction@[2][]%
{\begin{equation}\ce{#2}%
		\ifx\@empty#1\@empty\else\label{#1}\fi%
		\reactiontag\end{equation}}
\newcommand\reaction@nonumber[1]%
{\begin{equation*}\ce{#1}\end{equation*}}
\newcommand\reaction%
{\@ifstar{\reaction@nonumber}{\reaction@}}
\makeatother

\makeatletter
\renewcommand*{\@fnsymbol}[1]{\ensuremath{\ifcase#1\or 1 \or* \or 3  \or  4 \else\@ctrerr\fi}}
\makeatother

%\newcommand{\changefont}{\fontsize{10}{11}\selectfont}
%\fancyhead[L]{\changefont \slshape \nouppercase \rightmark} %section
%\fancyhead[R]{\changefont \slshape \nouppercase \leftmark} %chapter
%\setlength{\headheight}{12.5pt}
%\newcommand{\HRule}{\rule{\linewidth}{0.5mm}}
%\pagestyle{fancy}

% Index
\usepackage{imakeidx}
\makeindex

\usepackage{upquote} %To make straight quotes work in pdf
\usepackage[pagebackref=false,colorlinks,linkcolor=blue,citecolor=blue,urlcolor=blue]{hyperref}

%\citestyle{plainnat}
%\bibliographystyle{ieeetr}
\bibliographystyle{unsrtnat}

% Numbering \subsubsection and show it in TOC
\setcounter{secnumdepth}{3}
\setcounter{tocdepth}{3}

\usepackage{xcolor}

\usepackage{color,soul}

\usepackage[version=4]{mhchem}

\usepackage{comment}

% -------------------- listings definition -------------------- %
\usepackage{listings}

\usepackage{accsupp}
\newcommand{\noncopynumber}[1]{%
	\BeginAccSupp{method=escape,ActualText={}}%
	#1%
	\EndAccSupp{}%
}

\colorlet{mygray}{black!4}
\definecolor{mygreen}{rgb}{0,0.4,0}
\colorlet{myred}{red!80!blue}

\lstdefinestyle{cpp}{
	language=C++,
	basicstyle=\ttfamily\footnotesize,
	keepspaces=true,
	columns=fixed,%fullflexible
	fontadjust=true,
	basewidth=0.5em,
	backgroundcolor=\color{mygray},
	tabsize=4,
	captionpos=b,
	frame=single,
	numbers=left,
	numberstyle=\tiny\noncopynumber,
	numbersep=5pt,
	breaklines=true,
	showstringspaces=false,
	keywordstyle=\color{blue},
	commentstyle=\color{mygreen},
	stringstyle=\color{myred}
}

\lstdefinestyle{OpenFOAMDict}{
	language=C++,
	basicstyle=\ttfamily\footnotesize,
	keepspaces=true,
	columns=fixed,%fullflexible
	fontadjust=true,
	basewidth=0.5em,
	backgroundcolor=\color{mygray},
	tabsize=4,
	captionpos=b,
	frame=single,
	numbers=left,
	numberstyle=\tiny\noncopynumber,
	numbersep=5pt,
	breaklines=true,
	showstringspaces=false,
	commentstyle=\color{mygreen},
}
% --------------------------------------------------------------- %



% ------ A hack to prevent line break in \altt environment -------- %
\makeatletter
\newcommand*{\textalltt}{}
\DeclareRobustCommand*{\textalltt}{%
	\begingroup
	\let\do\@makeother
	\dospecials
	\catcode`\\=\z@
	\catcode`\{=\@ne
	\catcode`\}=\tw@
	\verbatim@font\@noligs
	\@vobeyspaces
	\frenchspacing
	\@textalltt
}
\newcommand*{\@textalltt}[1]{%
	#1%
	\endgroup
}
\makeatother
% --------------------------------------------------------------- %



% ------------------ Nomenclature definition ------------------- %
\usepackage{siunitx}
\sisetup{inter-unit-product=\ensuremath{{}\cdot{}},per-mode=symbol}
\usepackage{nomencl}
\usepackage{ifthen}
\renewcommand\nomgroup[1]{%				C - Constants
	\ifthenelse{\equal{#1}{C}}{%
		\vspace{0.5cm}%
		\item[\textbf{Constants}]}{%
		\ifthenelse{\equal{#1}{A}}{%
			\vspace{0.5cm}%
			\item[\textbf{Acronyms}]}{%              A - Acronyms
			\ifthenelse{\equal{#1}{E}}{%
				\vspace{0.5cm}%
				\item[{\textbf{English symbols}}]}{%           E - English
				\ifthenelse{\equal{#1}{G}}{%
					\vspace{0.5cm}
					\item[{\textbf{Greek symbols}}]}{%           G - Greek
					\ifthenelse{\equal{#1}{S}}{%
						\vspace{0.5cm}%	
						\item[\textbf{Superscripts}]}{%            S - Superscripts
						\ifthenelse{\equal{#1}{U}}{%
							\vspace{0.5cm}%	
							\item[\textbf{Subscripts}]}{%              U - Subscripts
							\ifthenelse{\equal{#1}{X}}{%
								\vspace{0.5cm}%	
								\item[\textbf{Other symbols}]}{%           X - Other Symbols
								{}}}}}}}}}
\renewcommand*{\nompreamble}{\markboth{\nomname}{\nomname}}
% \newcommand{\nomunit}[1]{\renewcommand{\nomentryend}{\hspace*{\fill}#1}}
\newcommand{\nomunit}[1]{\renewcommand{\nomentryend}{\dotfill#1}}
\makenomenclature
% --------------------------------------------------------------- %



\usepackage{titlesec}

\setcounter{secnumdepth}{4}

\titleformat{\paragraph}
{\normalfont\normalsize\bfseries}{\theparagraph}{1em}{}
\titlespacing*{\paragraph}
{0pt}{3.25ex plus 1ex minus .2ex}{1.5ex plus .2ex}


% Define a command to reset counters and add 'S' prefix for Supplementary section
\newcommand{\beginsupplement}{%
	% Figures
	\setcounter{figure}{0}%
	\renewcommand{\thefigure}{S\arabic{figure}}%
	% Tables
	\setcounter{table}{0}%
	\renewcommand{\thetable}{S\arabic{table}}%
	% Equations
	\setcounter{equation}{0}%
	\renewcommand{\theequation}{S\arabic{equation}}%
	% Sections
	\setcounter{section}{0}%
	\renewcommand{\thesection}{S\arabic{section}}%
	% Pages
	\setcounter{page}{1}%
	\renewcommand{\thepage}{S.\arabic{page}}%
	% Hyperref compatibility
	\renewcommand{\theHfigure}{S\arabic{figure}}%
	\renewcommand{\theHtable}{S\arabic{table}}%
	\renewcommand{\theHequation}{S\arabic{equation}}%
	\renewcommand{\theHsection}{S\arabic{section}}%
	%\renewcommand{\theHpage}{S.\arabic{page}}%
}

%-----------------------colored citation---------------------
\newcommand{\citeColored}[2]{%
	{\hypersetup{citecolor=#1}%
		\cite{#2}}%
}